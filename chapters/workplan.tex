%!TEX root = ../index.tex

\newpage
\section{Arbeits Planung} % (fold)
\label{sec:arbeits_planung}

\subsection{Marktanalyse und Platformevaluation} % (fold)
\label{sub:marktanalyse_und_platformevaluation}
Task-Owner: Marc Egli\\
In der Marktanalyse und Platformevaluation wird bestimmt was die Anforderungen an die Verwendbarkeit der Applikation und die dazu benötigte Platform sein soll. In der Marktanalyse zeigt sich welche Umgebungen und Phones in der jetzigen Zeit unterstützt werden müssen um einen Grossteil der Smartphone-Benutzer zu erreichen. Mit der Platformevaluation wird bestimmt mit welcher Technologie in gegebener Zeit ein möglichst gutes Resultat erzielt werden kann.
% subsection marktanalyse_und_platformevaluation (end)

\subsection{Individuelle Einrichtung der Applikationsplatform} % (fold)
\label{sub:individuelle_einrichtung_der_applikationsplatform}
Task-Owner: Marc Egli und Tarik Azarnait\\
Marc Egli besitzt ein Iphone 4S Tarik Azarnait besitzt ein Galaxy II S. Es war von Anfang an klar, das jeder die Applikation auf seinem Smartphone laufen lassen möchte. In diesem Task soll jeder auf seinem Computer die Platform einrichten und kennenlernen, damit man für die weitere Entwicklung vorbereitet ist. 
% subsection individuelle_einrichtung_der_applikationsplatform (end)

\subsection{Frontend Entwicklung} % (fold)
\label{sub:frontend_entwicklung}
Task-Owner: Tarik Azarnait\\
Die Frontend Entwicklung beinhaltet Design, Funktionalitätsimplementation, Testing des Applikationsteils, der auf dem Smartphone zu liegen kommt. Weil bei diesem Task schon entschieden war das PhoneGap benutzt werden soll, konnte dieser Task einfach in HTML und Java Script programmiert werden um auf Android und iOS zum laufen zu kommen. 
% subsection frontend_entwicklung (end)

\subsection{Backend Entwicklung} % (fold)
\label{sub:backend_entwicklung}
Task-Owner: Marc Egli\\
Das Resultat der Backend Entwicklung soll ein Applikationsserver sein, der die entsprechenden Spieldaten verwaltet und den einzelnen Spielteilnehmer zukommen lässt. Die Daten sollen während der Initialisierung der Applikation vom Backend abgefragt werden.
% subsection backend_entwicklung (end)

\subsection{Dokumentation} % (fold)
\label{sub:dokumentation}
Task-Owner: Marc Egli und Tarik Azarnait\\
Dokumentation beinhaltet das Beschreiben der ganzen Arbeit, Erläuterung der Idee, des Konzeptes und der Lösung, Dokumentation der Applikation und des Backends. Alles soll in diesem Dokument vereint und als Seminararbeit abgegeben werden. 
% subsection dokumentation (end)

% section arbeits_plan (end)