\newpage
\section{Fuchsjagt Sport} % (fold)
\label{sec:Fuchsjagt Sport}

Unter Fuchsjagd versteht man im Amateurfunk eine Outdoor-Sportart, die an einen Orientierungslauf erinnert. Dabei sind die Teilnehmer mit tragbaren Funkempfänger ausgestattet, mit denen sie die verschiedenen Anlaufstellen orten können. An diesen zu suchenden Stellen werden Rundstrahlantennen versteckt, welche Signale aussenden.

Dies schafft die Grundlage für den Amateurfunk Orientierungslauf, wobei ein Teilnehmer um das Ziel zu erreichen alle Stationen suchen und anlaufen muss. Als weitere Hilfsmittel hat er zusätzlich eine Karte und einen Kompass. Damit kann er die Ziele anhand von der Signalstärke ungefähr ermitteln und sich die Route planen, die nicht vom Spiel vorgegeben ist. 

\subsection{Geschichte} % (fold)
\label{sub:Geschichte}

Schon in den Zwanzigerjahren begann man Mittels Richtfunkgeräten mit der Fuchsjagd. In den jungen Jahren des Sports spielte man auf einem See wo bei der Fuchs (Sender) in einem Schiff auf ein bestimmtes Ziel zu bewegte. Das Ziel war das Schiff zu finden bevor es sein Ziel erreichte. Durch die Wellen auf dem Wasser wurde die Peilung jedoch sehr erschwert und so wurde der Fuchs nur mit Glück gefunden.

In den Sechzigerjahren fand das Spiel dann in der heutigen Ausprägung auf dem Land statt, wobei die Sendeanlage in einem Auto in einem Wald versteckt wurde. 
% subsection Geschichte (end)

\subsection{Ausprägungen} % (fold)
\label{sub:Ausprägungen}

Ausser der Fuchsjagd mit dem finden der Sender mit Hilfe von Empfänger, Karte und Kompass in möglichst kurzer Zeit, gibt es noch andere Ausprägungen:

\subsubsection{Foxoring} % (fold)
\label{ssub:Foxoring}
Ein Orientierungslauf mit Zielbereichen, die auf der Karte eingetragen und bekannt sind. In einem solchen Zielkreis angekommen muss der Mitstreiter das genaue Ziel jedoch mithilfe seines Richtfunkempfängers orten.
% subsubsection Foxoring (end)

\subsubsection{Fuchswanderung} % (fold)
\label{ssub:Fuchswanderung}
Eine Fuchsjagd, welche nicht auf Zeit gespielt wird und die Teilnehmer über einen schöne Wanderung in der Natur führt.
% subsubsection Fuchswanderung (end)

\subsubsection{Grossraum-Fuchsjagd} % (fold)
\label{ssub:Grossraum-Fuchsjagd}
Dabei wird ein Fuchs in einem grossen Gebiet versteckt. Für die Teilnahme am Wettbewerb, kann man seine einfach die auf Grund von Peilung geschätzte Position des Fuchs einschicken, wobei die nächste Schätzung gewinnt. Weiter nehmen am Wettbewerb auch mobile Sucher teil, welche versuchen den Fuchs tatsächlich zu finden.
% subsubsection Grossraum-Fuchsjagd (end)
% subsection Ausprägungen (end)

% section Fuchsjagt Sport (end)
