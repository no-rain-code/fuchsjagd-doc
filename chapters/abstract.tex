%!TEX root = ../index.tex

\newpage
\section{Abstract} % (fold)
\label{sec:Abstract}
Es soll das Spiel “Fuchsjagd” welches normalerweise mit Peilsendern und Richtantennen gespielt wird, für Smartphones entwickelt werden welche mit Hilfe von GPS und Kompass den Gebrauch von Funkequipment erübrigen.
Dadurch kann dieses Spiel auch ohne eine Amatuerfunklizenz gespielt werden. Die Applikation sollte auf möglichst vielen Platformen laufen, so dass jeder Teilnehmer mit seinem eigenen Smartphone mitspielen kann.

Grundsätzlich soll die Seminararbeit als eine Einführung in die Smartphone Entwicklung dienen. Da die Anwendung auf möglichst vielen Smartphones benutzbar sein soll, muss dafür eine Technologie (z.B Phonegap) evaluiert und gebraucht werden, welche dies ermöglicht. Als Weiterführung könnte man die Implementation, Vor- und Nachteile auf den spezifischen nativen Technologien prüfen.  

Ziel ist es das Spiel spielbar in einer Smartphone Anwendung implementiert zu haben. Implementationsspezifisch zu evaluierende Punkte sind die Konfiguration für die Spielrunde (z.B. zu suchende Koordinaten setzen), welche direkt gemacht wird im Phone oder über eine administrierende Applikation an die Teilnehmer gesendet werden kann, und das Monitoring, mit dem man möglicherweise den ganzen Spielverlauf auf einer administrierenden Applikation mitverfolgen und nachvollziehen kann.
% section Abstract (end)

Thema: Implementation einer Smartphone Anwendug für das Funkspiel Fuchjagt

\subsubsection{Ausgangslage} % (fold)
\label{ssub:Ausgangslage}
Das Spiel Fuchsjagt benötigt teure und komplizierte Funkgeräte, um mit einem Peilsender und Richtantennen bestimmte Orte zu finden. Es soll eine Anwendung entwickelt werden, welche das Fuchsjagt spielen für einfache Smartphonebesitzer möglich macht.
% subsubsection Ausgangslage (end)

\subsubsection{Ziel der Arbeit} % (fold)
\label{ssub:Ziel der Arbeit}
Das Ziel der Arbeit ist die Entwicklung einer Anwendung, welche das Spielen der Fuchsjagt mittels einem Smartphone ermöglicht. Die Anwendung soll auf möglichst vielen verschiedenen Smartphones brauchbar sein.
% subsubsection Ziel der Arbeit (end)

\subsubsection{Aufgabenstellung} % (fold)
\label{ssub:Aufgabenstellung}
Einarbeiten in das Feld der Smartphone Entwicklung.
Evaluieren einer geeigneten Technologie, um das Requirement der mobilität auf vielen verschiedenen Geräte zu erreichen (z.B. Phonegap)
Entwicklung der Anwendung mit Design, Implementation, Testing auf Iphone 4 und Android 2.3.
Weiterführend:
Aufzeigen der Vor- und Nachteile von nativen Implementationen
Verbesserte Konfiguration über eine administrierende Applikation
Monitoring der Andwender während dem gebraucht der Andwendung.
% subsubsection Aufgabenstellung (end)

\subsection{Erwartete Resultate} % (fold)
\label{sub:Erwartete Resultate}
Eine laufende Smartphoneanwendung die das Spielen von Fuchsjagt ermöglicht.
Eine Dokumentation welche folgende Punte enthält:
\begin{itemize}
    \item Evaluation einer Technologie, welche auf vielen Smartphones anwendbar ist.
    \item Dokumentation der Applikation: Design, Implementation, Testing und Gebrauch der Anwendung.
    \item Weiterführende Themen: Evaluation von nativen Implementationen, Konfiguration und Monitoring einer administrierenden Applikation
\end{itemize}

% subsection Erwartete Resultate (end)

\subsection{Geplante Termine} % (fold)
\label{sub:Geplante Termine}

\begin{tabular}{|l|l|}
\hline
                   Kickoff & 14.10.2011\\
\hline
  Aufgabenstellung erfasst & 28.10.2011\\
\hline
             Abgabe Teaser & 30.11.2011\\
\hline
            Arbeitstreffen &  7.12.2011\\
\hline
Abgabe Schriftliche Arbeit &  11.1.2012\\
\hline
             Präsentation &  18.1.2012\\
\hline
\end{tabular}

% subsection Geplante Termine (end)
