%!TEX root = ../index.tex

\newpage
\section{Abstract} % (fold)
\label{sec:Abstract}
Für Smartphones soll das Spiel "`Fuchsjagd"' entwickelt werden, welches normalerweise mit Peilsendern und Richtantennen gespielt wird. Mit Hilfe von GPS und Kompass erübrigt sich der Gebrauch von Funkequipment.
Dadurch kann dieses Spiel auch ohne eine Amateurfunklizenz gespielt werden. Die Applikation sollte auf möglichst vielen Plattformen laufen, so dass jeder Teilnehmer mit seinem eigenen Smartphone mitspielen kann.

Grundsätzlich soll die Seminararbeit als eine Einführung in die Smartphone Entwicklung dienen. Da die Anwendung auf möglichst vielen Smartphones benutzbar sein soll, muss dafür eine Technologie (z.B. PhoneGap) evaluiert und gebraucht werden, welche dies ermöglicht. Als Weiterführung könnte man die Implementation, Vor- und Nachteile auf den spezifischen nativen Technologien prüfen.  

Ziel ist es, das Spiel spielbar in einer Smartphone Anwendung implementiert zu haben. Für das Design zu evaluierende Punkte sind die Konfiguration für die Spielrunde (z.B. zu suchende Koordinaten setzen), welche direkt im Smartphone gemacht wird oder über eine administrierende Applikation an die Teilnehmer gesendet werden kann, und das Monitoring, mit dem man möglicherweise den ganzen Spielverlauf auf einer administrierenden Applikation mitverfolgen und nachvollziehen kann.
% section Abstract (end)

Thema: Implementation einer Smartphone Anwendung für das Funkspiel Fuchsjagd

\subsubsection{Ausgangslage} % (fold)
\label{ssub:Ausgangslage}
Das Spiel Fuchsjagd benötigt teure und komplizierte Funkgeräte, um mit einem Peilsender und Richtantennen bestimmte Orte zu finden. Es soll eine Anwendung entwickelt werden, welche das Spiel Fuchsjagd für einfache Smartphone-Besitzer ermöglicht.
% subsubsection Ausgangslage (end)

\subsubsection{Ziel der Arbeit} % (fold)
\label{ssub:Ziel der Arbeit}
Das Ziel der Arbeit ist die Entwicklung einer Anwendung, welche das Spielen der Fuchsjagd mittels einem Smartphone ermöglicht. Die Anwendung soll auf möglichst vielen verschiedenen Smartphones brauchbar sein.
% subsubsection Ziel der Arbeit (end)

\subsubsection{Aufgabenstellung} % (fold)
\label{ssub:Aufgabenstellung}
Einarbeiten in das Feld der Smartphone Entwicklung.
Evaluieren einer geeigneten Technologie, um das Requirement der Mobilität auf vielen verschiedenen Geräte zu erreichen (z.B. PhoneGap)
Entwicklung der Anwendung mit Design, Implementation, Testing auf Iphone 4 und Android 2.3.
Weiterführend:
Weiterführend: Aufzeigen der Vor- und Nachteile von nativen Implementationen und verbesserter Konfiguration der Anwender während des Gebrauchs über eine administrierende Monitoring Applikation.
% subsubsection Aufgabenstellung (end)

\subsection{Erwartete Resultate} % (fold)
\label{sub:Erwartete Resultate}
Eine laufende Smartphoneanwendung die das Spielen von Fuchsjagd ermöglicht.
Eine Dokumentation welche folgende Punkte enthält:
\begin{itemize}
    \item Evaluation einer Technologie, welche auf vielen Smartphones anwendbar ist.
    \item Dokumentation der Applikation: Design, Implementation, Testing und Gebrauch der Anwendung.
    \item Weiterführende Themen: Evaluation von nativen Implementationen, Konfiguration und Monitoring einer administrierenden Applikation
\end{itemize}

% subsection Erwartete Resultate (end)

\subsection{Geplante Termine} % (fold)
\label{sub:Geplante Termine}

\begin{tabular}{|l|l|}
\hline
                   Kickoff & 14.10.2011\\
\hline
  Aufgabenstellung erfasst & 28.10.2011\\
\hline
             Abgabe Teaser & 30.11.2011\\
\hline
            Arbeitstreffen &  07.12.2011\\
\hline
Abgabe Schriftliche Arbeit &  11.01.2012\\
\hline
             Präsentation &  18.01.2012\\
\hline
\end{tabular}

% subsection Geplante Termine (end)
